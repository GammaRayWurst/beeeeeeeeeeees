\documentclass[aps,prl,preprint,groupedaddress,twocolumn]{revtex4-1}



\begin{document}


\title{BEEEEEEEEEEEEEEEES}

% repeat the \author .. \affiliation  etc. as needed
% \email, \thanks, \homepage, \altaffiliation all apply to the current
% author. Explanatory text should go in the []'s, actual e-mail
% address or url should go in the {}'s for \email and \homepage.
% Please use the appropriate macro foreach each type of information

% \affiliation command applies to all authors since the last
% \affiliation command. The \affiliation command should follow the
% other information
% \affiliation can be followed by \email, \homepage, \thanks as well.
\author{Julia???}
%\email[]{Your e-mail address}
%\homepage[]{Your web page}
%\thanks{}
%\altaffiliation{}
\affiliation{}
\author{Sin\'ead Walsh}
\affiliation{Max Planck Institute for Gravitational Physics (Albert Einstein Institute)}

\author{Sylvia J. Zhu}
\affiliation{Max Planck Institute for Gravitational Physics (Albert Einstein Institute)}

%Collaboration name if desired (requires use of superscriptaddress
%option in \documentclass). \noaffiliation is required (may also be
%used with the \author command).
%\collaboration can be followed by \email, \homepage, \thanks as well.
%\collaboration{}
%\noaffiliation

\date{\today}

\begin{abstract}
BEEEEEEEES
\end{abstract}

% insert suggested PACS numbers in braces on next line
\pacs{}
% insert suggested keywords - APS authors don't need to do this
%\keywords{}

%\maketitle must follow title, authors, abstract, \pacs, and \keywords
\maketitle

% body of paper here - Use proper section commands
% References should be done using the \cite, \ref, and \label commands
\section{Introduction}

Bees are a common phenomena in the bucolic countryside. They are easily
recognized by their abdominal stripes of alternating yellow and black, their
buzzing presence around flowering plants, and their painful defense
mechanisms. Some bees produce honey, which is enjoyed by bears (such as
brown bears, black bears, and Winnie the Pooh). Due to their ubiquitous
presence in nature, they have infiltrated the collective consciousness:
Bees appear in children's songs (e.g., a baby bumblebee is successively
brought home, squished up, licked up, thrown up, wiped up, and wring'd out
by a narrator), alternative '90s American rock songs (a dancing
anthropomorphic bee girl in a music video), and terrible animated movies
with unimaginative names that are thinly-veiled vehicles for aging comedians
past their prime.

Unlike bees, most gravitational wave physicists (GWPs) do not produce delicious amber
nectar by flitting 'twixt flowers. However, contrary to popular thought,
GWPs can at times be found frolicking outdoors and in close proximity to these
same flowers. A GWP who encounters a bee is sometimes judged by the latter
to be an enemy, and can be subjected to a stinging attack (known as a
``bee sting'').

In this paper, we present a population study to elucidate the
GWP-bee relationship. We will use ``bee'' as shorthand
to mean ``bee, or hornet, or wasp, or any similar stinging flying insect.''


\section{Observations}

Observations took place over a period of two days, during a portion of
the LVC meeting in Pasadena, California in March of 2016 (GPS = 1142125217
to 1142308817 s). We approached individuals in our vicinity and verbally
inquired about their personal bee-related history. Due to the fallibility
of human memory, we chose to use a few broad response categories rather
than ask for specific numbers. The categories we chose were:
\begin{itemize}
\item ``None'': The subject has never been stung by a bee before.
\item ``A few'': The subject has been stung by a bee less than or equal
  to three times throughout their life.
\item ``A lot'': The subject has been stung by a bee more than three times
  but less than or equal to ten times throughout their life.
\item ``Many'': The subject has been stung by a bee more than ten times
  in their life.
\end{itemize}

The observational period occurred in essentially two stages. During the
first stage (approximately two hours), we only approached subjects with
whom we had preestablished acquaintances. However, as the night wore on
and we reached the end of our acquaintance list, we began to approach
people who were unknown to us. We note that, during this second stage,
we avoided approaching senior scientists.


\section{Analysis}

\textbf{(some bee-s)}

Five GWPs reported having been stung by a bee ``many'' times during
their lifetimes, defined as ``ten or more times.'' Out of these five,
four had a close family member or friend with a bee farm. Without further
information, we can only speculate that the fifth GWP is a terrible
person and an enemy to all wildlife.


\section{Discussion}

\textbf{(some more bee-s)}

\subsection{Study limitations and caveats}

We conducted a large portion of this study during a
reception for which ample food had been
promised but never procured. Because of this, many reception attendees
had been imbibing alcohol on empty or near-empty stomachs, which could
have affected their memory and their responses.

During the data collection process, one of the data takers was
coincidentally wearing a shirt with gold and black stripes. This could
have ...

As we previously mentioned, during stage two of our data collection process,
we preferentially approached scientists who appeared to be more junior.
Therefore, our sample collection is biased toward younger and/or
immature-presenting GWPs.

\section{Conclusion}

In this work, we have presented an observational study on the common
experience of bees by GWPs. As a next step, the authors will conduct
injection studies by releasing bees into various conference rooms.

We thank the LVC for providing us with an ample sample of GWPs.


% Put \label in argument of \section for cross-referencing
%\section{\label{}}
%\subsection{}
%\subsubsection{}

% If in two-column mode, this environment will change to single-column
% format so that long equations can be displayed. Use
% sparingly.
%\begin{widetext}
% put long equation here
%\end{widetext}

% figures should be put into the text as floats.
% Use the graphics or graphicx packages (distributed with LaTeX2e)
% and the \includegraphics macro defined in those packages.
% See the LaTeX Graphics Companion by Michel Goosens, Sebastian Rahtz,
% and Frank Mittelbach for instance.
%
% Here is an example of the general form of a figure:
% Fill in the caption in the braces of the \caption{} command. Put the label
% that you will use with \ref{} command in the braces of the \label{} command.
% Use the figure* environment if the figure should span across the
% entire page. There is no need to do explicit centering.

% \begin{figure}
% \includegraphics{}%
% \caption{\label{}}
% \end{figure}

% Surround figure environment with turnpage environment for landscape
% figure
% \begin{turnpage}
% \begin{figure}
% \includegraphics{}%
% \caption{\label{}}
% \end{figure}
% \end{turnpage}

% tables should appear as floats within the text
%
% Here is an example of the general form of a table:
% Fill in the caption in the braces of the \caption{} command. Put the label
% that you will use with \ref{} command in the braces of the \label{} command.
% Insert the column specifiers (l, r, c, d, etc.) in the empty braces of the
% \begin{tabular}{} command.
% The ruledtabular enviroment adds doubled rules to table and sets a
% reasonable default table settings.
% Use the table* environment to get a full-width table in two-column
% Add \usepackage{longtable} and the longtable (or longtable*}
% environment for nicely formatted long tables. Or use the the [H]
% placement option to break a long table (with less control than 
% in longtable).
% \begin{table}%[H] add [H] placement to break table across pages
% \caption{\label{}}
% \begin{ruledtabular}
% \begin{tabular}{}
% Lines of table here ending with \\
% \end{tabular}
% \end{ruledtabular}
% \end{table}

% Surround table environment with turnpage environment for landscape
% table
% \begin{turnpage}
% \begin{table}
% \caption{\label{}}
% \begin{ruledtabular}
% \begin{tabular}{}
% \end{tabular}
% \end{ruledtabular}
% \end{table}
% \end{turnpage}

% Specify following sections are appendices. Use \appendix* if there
% only one appendix.
%\appendix
%\section{}

% If you have acknowledgments, this puts in the proper section head.
%\begin{acknowledgments}
% put your acknowledgments here.
%\end{acknowledgments}

% Create the reference section using BibTeX:
\bibliography{basename of .bib file}

\end{document}
%
% ****** End of file apstemplate.tex ******

